\documentclass{beamer}
\hypersetup{colorlinks=true, linkcolor=white, urlcolor=red}
\usetheme{Copenhagen}
\title{Presentation Title}
\subtitle{Subtitle}
\author{u/androgynyjoe}
\institute{Reddit}

\begin{document}
\frame{\titlepage}

\section{Information}
\subsection{Instructions}

\begin{frame}
\frametitle{Instructions}

If you're on a chromebook and you've never used a slideshow made by Beamer before then here's what you do.
\begin{enumerate}
\item Open this pdf document in Chrome (you've probably already done that).
\item Make sure that the document is set to ``fit to width.''  For me, if you mouse over to the bottom right-hand corner there are three buttons and I click the top one.  (Depending on your aspect ratio it might not fit the width but exactly one frame should be on the screen at a time.)
\item Put chrome in fullscreen mode.  On my keyboard that is the fourth button from the left on the top row (to the right of the refresh key).
\item Use the left and right arrow keys to navigate through the frames.
\end{enumerate}
\end{frame}

\subsection{About \LaTeX\ and Beamer}

\begin{frame}
\frametitle{What is this sorcery?}
If, for whatever reason, you're interested in learning how to make presentations like this, the tool is called \href{https://www.overleaf.com/learn/latex/Beamer}{Beamer}.  It is one of the many things that can be typeset and formatted using a software package called \href{https://www.latex-project.org/about/}{\LaTeX}.  It is used in many technical professions (especially those that frequently need to typeset mathematics) to create beautiful, professional documents.

\bigskip

However, there is a bit of a learning curve and it's not for everyone.  If you don't have a specific reason that you want to learn \LaTeX\ and/or Beamer then you probably shouldn't bother.  Whatever you're using now is perfectly fine.  :-)
\end{frame}

\begin{frame}
Why mathematics specifically?  Things like the following equations can be fairly difficult to do in traditional word processors but they are trivially easy in \LaTeX\ (once you know how).
\bigskip
\[ \langle \alpha , \beta \rangle = \frac{1}{\vert G \vert} \sum_{g \in G} \alpha(g) \overline{\beta(g)} \]
\bigskip
{\scriptsize
\begin{equation*}
\begin{split}
&\iint_\Sigma \Bigg( \left( \frac{\partial R}{\partial y} - \frac{\partial Q}{\partial z} \right) \mathrm{d}y\,\mathrm{d}z + \left( \frac{\partial P}{\partial z} + \frac{\partial R}{\partial x} \right) \mathrm{d}z\,\mathrm{d}x + \left( \frac{\partial Q}{\partial x} + \frac{\partial P}{\partial y} \right) \mathrm{d}x\,\mathrm{d}y   \Bigg) \\
&\hspace{22em} = \oint_{\partial\Sigma} (P\,\mathrm{d}x + Q\,\mathrm{d}y + R\,\mathrm{d}z)
\end{split}
\end{equation*}
}
\bigskip
\[ f(x) = \lim_{N\to\infty}\sum_{n=0}^N \frac{(x-a)^n}{n!} \left[\frac{\mathrm{d}f}{\mathrm{d}x}\right\vert_{x=a} \]
%\[ \mathrm{dim}_k \mathrm{Ext}_R^i (M,k) \geq \mathrm{dim}_k \mathrm{Ext}_R^i (k,k) = \binom{n}{k}  \]
\end{frame}

\section{Random Nonsense}
\subsection{Name of a section}

\begin{frame}
I'm just going to add some more random frames to fill out a full presentation.
\end{frame}

\begin{frame}
This frame is my favorite frame.  I don't know why.
\end{frame}

\subsection{Another Section}

\begin{frame}
\begin{block}{This must be important}
It's got a box around it and everything.
\end{block}
\pause
\begin{itemize}
\item Look at the bullet points.\pause
\item Look at those simple, non-existent frame transitions.\pause
\item ooooooooh\pause
\item aaaaaaaah
\end{itemize}
\end{frame}

\subsection{The End}

\begin{frame}
Now it's over.  How sad.

\bigskip

Thank you for coming to my TED talk.  I worked really hard on it.
\end{frame}
\end{document}
%%% Local Variables:
%%% mode: latex
%%% TeX-master: t
%%% End:
